


% Define MATLAB code style for listings package
\definecolor{mygreen}{rgb}{0,0.6,0}
\definecolor{mygray}{rgb}{0.5,0.5,0.5}
\definecolor{mymauve}{rgb}{0.58,0,0.82}

\lstset{
    language=Matlab,
    keywordstyle=\color{blue},
    commentstyle=\color{mygreen},
    stringstyle=\color{mymauve},
    basicstyle=\ttfamily\small,
    breaklines=true,
    numbers=left,
    numberstyle=\tiny\color{mygray},
    stepnumber=1,
    numbersep=10pt,
    tabsize=4,
    showspaces=false,
    showstringspaces=false
}



\title{MATLAB Script: ramsey.m}
\author{Matt Murillo}
\date{}

\maketitle

\section*{MATLAB Code}
\begin{lstlisting}
%%%%%%%%%%%%%%%%%%%%%%%%%%%%%%%%%%%%%%%%%%%%%%%%%%%%%%%%%%%%%%%%%%%%%%%%%%%
% MATLAB Script: ramsey.m
%
% Author: Matt Murillo


%%%%%%%%%%%%%%%%%%%%%%%%%%%%%%%%%%%%%%%%%%%%%%%%%%%%%%%%%%%%%%%%%%%%%%%%%%%

clear all;
close all;
clc;

%% 1. DEFINE PARAMETERS

p = define_parameters();

%% 2. INITIALIZE GRID POINTS

t = linspace(p.tmin, p.tmax, p.I)';
dt = (p.tmax-p.tmin)/(p.I-1);

%% 3. STEADY STATES Analytical
% kss = ((rho + theta*g)/(alpha*A))^(1/(alpha-1))
% css = kss * (rho + (theta*g) - (alpha*(n + g)))/alpha

kss = ((p.rho + p.theta * p.g) / (p.alpha * p.A))^(1 / (p.alpha - 1));
css = kss * (p.rho + (p.theta*p.g) - (p.alpha*(p.n + p.g)))/p.alpha;

%% 4. STEADY STATE FSOLVE


diff = @(x) [p.f(x(1)) - x(2) - (p.n + p.g) * x(1); p.f_prime(x(1)) - p.rho - p.theta * p.g]; 

k0 = 10;
c0 = 10;

x0 = [k0; c0];  % Initial guess for k and c

options = optimoptions('fsolve', 'Display', 'off');
x = fsolve(diff, x0, options);

% Steady state k and c and interest rate
kss_N = x(1);
css_N = x(2);

%% 5. SHOOTING ALGORITHM

% 4-1. Find the initial value of consumption that satisfies terminal boundary condition

tic;
% Objective function that calculates the difference between terminal capital k(T) and steady-state kss
% Note: k(T)-kss is a function of c0
diff = @(c0) terminal_condition(c0, p.k0, kss, p.f, p.f_prime, p.rho, p.theta, p.g, p.n, dt, p.I);

% Guess an initial value of consumption
c0_guess = 1;

% Use fsolve to find the initial consumption c0 that makes k(T) = k_ss
% Note: X = fsolve(FUN, X0, options) starts at the matrix X0 and tries to solve the equations in FUN.
% Set OPTIONS = optimoptions('fsolve','Algorithm','trust-region'), and then pass OPTIONS to fsolve.
options = optimoptions('fsolve', 'TolFun', p.tol, 'Display', 'iter');
c0 = fsolve(diff, c0_guess, options);

% 4-2. Forward simulate with the updated initial consumption

[k, c] = forward_simulate(c0, p.k0, p.f, p.f_prime, p.rho, p.theta, p.g, p.n, dt, p.I);
toc;

% 5-1. Evolution of capital and consumption and interest rate
rt = p.f_prime(k)

figure;
subplot(3,1,1);
plot(t, k, 'r-', 'LineWidth', 2);
xlabel('Time'); ylabel('Capital k(t)');
title('Capital Accumulation over Time');

subplot(3,1,2);
plot(t, c, 'b-', 'LineWidth', 2);
xlabel('Time'); ylabel('Consumption c(t)');
title('Consumption Growth over Time');

subplot(3,1,3);
plot(t, rt, 'b-', 'LineWidth', 2);
xlabel('Time'); ylabel('Interest rate r(t)');
title('Interest rate over Time');

saveas(gcf, "fig1.png")

%% 6 Newton's Method 

% Initial guess
k0 = 10;
c0 = 10;

b0 = [k0;c0]

for i = 1: 1000
kt= b0(1)
ct= b0(2)

J_inv = inv([(p.alpha * p.A * kt^(p.alpha-1)) - (p.n + p.g), -1; (ct*p.alpha*(p.alpha-1)*p.A*kt^(p.alpha-2))/p.theta, (p.alpha*p.A*kt^(p.alpha-1)-p.rho-(p.theta*p.g))/p.theta])

F = [p.A*kt^(p.alpha)-ct-((p.n + p.g)*kt); ct*((p.alpha*p.A*kt^(p.alpha-1)-p.rho-(p.theta*p.g))/p.theta)]

b1 = b0 - J_inv*F

if norm(b1-b0) < p.tol
        break;
end

b0 = b1

end


function p = define_parameters()

% This function defines the parameters needed for the ramsey.m script

%% Economic Parameters

    % Discount rate
    p.rho = 0.03;

    % Inverse of IES 
    p.theta = 1;

    % Technology growth rate
    p.g = 0.02;

    % Population growth rate
    p.n = 0.02;

    % Capital share
    p.alpha = 1/3;

    % TFP
    p.A = 1;

%% Economic Functions
    
    % Production function
    p.f = @(k) p.A * k.^p.alpha;

    % MPK
    p.f_prime = @(k) p.alpha * p.A * k.^(p.alpha - 1);  

%% Boundary Conditions

    % Initial capital
    p.k0 = 10;

%% Grid Paramters

    p.tmin = 0;
    p.tmax = 100;

    % The number of time steps
    p.I = 300; 

%% Newton Method Tuning Parameters
    
    % Tolerance for Newton's method
    p.tol = 1e-6;  

    % Maximum iterations for Newton's method
    % p.maxit = 1000;  

end

function [k, c] = forward_simulate(c0, k0, f, f_prime, rho, theta, g, n, dt, I)

% This function solves the two differential equations using forward simulation.

    % Pre-allocate arrays for solution

    k = zeros(I, 1);  
    c = zeros(I, 1);  
    k(1) = k0;
    c(1) = c0;
    
    for i = 1:I-1
        
        % Euler equation for consumption growth: (c(i+1)-c(i))/dt = c(i)*(f'(k(i)-rho-theta*g)/theta
        c(i+1) = c(i) + dt * (f_prime(k(i)) - rho - theta * g) / theta * c(i);
        
        % Capital accumulation equation:(k(i+1)-k(i))/dt = f(k(i))-c(i)-(n+g)k(i)
        k(i+1) = k(i) + dt * (f(k(i)) - c(i) - (n + g) * k(i));
    end

end

function difference = terminal_condition(c0, k0, kss, f, f_prime, rho, theta, g, n, dt, I)

% This function calculates the difference between terminal capital k(T) and
% steady-state kss.

    [k, ~] = forward_simulate(c0, k0, f, f_prime, rho, theta, g, n, dt, I);
    k_T = k(end);  % Terminal capital k(T)
    
    % The difference between terminal k(T) and steady-state k_ss
    difference = k_T - kss;  
end

\end{lstlisting}


