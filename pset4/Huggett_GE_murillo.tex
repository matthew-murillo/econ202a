


% Define MATLAB code style for listings package
\definecolor{mygreen}{rgb}{0,0.6,0}
\definecolor{mygray}{rgb}{0.5,0.5,0.5}
\definecolor{mymauve}{rgb}{0.58,0,0.82}

\lstset{
    language=Matlab,
    keywordstyle=\color{blue},
    commentstyle=\color{mygreen},
    stringstyle=\color{mymauve},
    basicstyle=\ttfamily\small,
    breaklines=true,
    numbers=left,
    numberstyle=\tiny\color{mygray},
    stepnumber=1,
    numbersep=10pt,
    tabsize=4,
    showspaces=false,
    showstringspaces=false
}



\title{MATLAB Script}
\author{Matt Murillo}
\date{}

\maketitle

\section*{MATLAB Code}
\begin{lstlisting}
%%%%%%%%%%%%%%%%%%%%%%%%%%%%%%%%%%%%%%%%%%%%%%%%%%%%%%%%%%%%%%%%%%%%%%%%%
% MATLAB Code: HJB_Huggett_GE
% 
% Author: Matt Murillo
% Date: Nov 7, 2024
%
% Description:
% This MATLAB script solves the general equilibrium of the Huggett model,
% finding the equilibrium interest rate that clears the bond market.
%
% Reference: Huggett_equilibrium_iterate.m by Benjamin Moll
%
% Code Structure:
% 1. DEFINE PARAMETERS
% 2. INITIALIZE GRID POINTS
% 3. PRE-ITERATION INITIALIZATION
% 4. VALUE FUNCTION ITERATION
% 5. KF EQUATION
% 6. GRAPHS
%%%%%%%%%%%%%%%%%%%%%%%%%%%%%%%%%%%%%%%%%%%%%%%%%%%%%%%%%%%%%%%%%%%%%%%%%

clear all;
close all;
clc;

%% 1. DEFINE PARAMETERS

p = define_parameters_GE();

%% 2. INITIALIZE GRID POINTS

a = linspace(p.amin, p.amax, p.I)';
da = (p.amax-p.amin)/(p.I-1);

aa = [a, a]; % I*2 matrix

        % %% 2-2. INITIALIZE GRID POINTS FOR INTEREST RATES
        % 
        % rgrid = linspace(p.rmin, p.rmax, p.Ir)';

%% 3. PRE-ITERATION INITIALIZATION

% 3-1. Construct the forward and backward differential operator 
% Df such that Df*V=dVf and Db such that Db*V=dVb

    Df = zeros(p.I, p.I);
    for i = 1:p.I-1
        Df(i,i) = -1/da; Df(i,i+1) = 1/da;
    end
    Df = sparse(Df);

    Db = zeros(p.I, p.I);
    for i = 2:p.I
        Db(i,i-1) = -1/da; Db(i,i) = 1/da;
    end
    Db = sparse(Db);

% 3-2. Construct A_switch matrix

    A_switch = [speye(p.I).*(-p.lambda(1)), speye(p.I).*p.lambda(1);
                speye(p.I).*p.lambda(2), speye(p.I).*(-p.lambda(2))];

%A_switch = zeros(2*I, 2*I);
%for i=1:I
%    A_switch(i,i) = -lambda(1);
%    A_switch(i,i+I) = lambda(1);
%    A_switch(i+I,i) = lambda(2);
%    A_switch(i+I,i+I) = -lambda(2);
%end

        %% 3-3. Guess an initial value of the interest rate

            r0 = 0.03;
            r_min = 0.01;
            r_max = 0.04;

        %% 3-4. Guess an initial value of the value function

            zz = ones(p.I, 1).*p.zz; % I*2 matrix
            
            % The value function of "staying put"
            r = r0;

            v0 = p.u(zz + r.*aa)./p.rho; 
            V = v0;

%% 4. VALUE FUNCTION ITERATION

% Initial guess

        % for nr=1:p.Nr
for nr=1:p.Nr

    
            r_r(nr) = r;
            rmin_r(nr) = r_min;
            rmax_r(nr) = r_max;
                               
            % Use the value function solution from the previous interest rate iteration 
            % as the initial guess for the next iteration
            if nr>1
               v0 = V_r(:,:,nr-1);
               V = v0;
            end

for n=1:p.maxit

    % 4-1. Compute the derivative of the value function 
    dVf = Df*V;
    dVb = Db*V;

    % 4-2. Boundary conditions
    dVb(1,:) = p.mu(zz(1,:) + r.*aa(1,:)); % a>=a_min is enforced (borrowing constraint)
    dVf(end,:) = p.mu(zz(end,:) + r.*aa(end,:)); % a<=a_max is enforced which helps stability of the algorithm

    I_concave = dVb > dVf; % indicator whether value function is concave (problems arise if this is not the case)

    % 4-3. Compute the optimal consumption
    cf = p.inv_mu(dVf);
    cb = p.inv_mu(dVb);
    
    % 4-4. Compute the optimal savings
    sf = zz + r.*aa - cf;
    sb = zz + r.*aa - cb;

    % 4-5. Upwind scheme
    If = sf>0;
    Ib = sb<0;
    I0 = 1-If-Ib;
    dV0 = p.mu(zz + r.*aa); % If sf<=0<=sb, set s=0

    dV_upwind = If.*dVf + Ib.*dVb + I0.*dV0;

    c = p.inv_mu(dV_upwind);

    % 4-6. Update value function: 
    % Vj^(n+1) = [(rho + 1/Delta)*I - (Sj^n*Dj^n+A_switch)]^(-1)*[u(cj^n) + 1/Delta*Vj^n]
    
    V_stacked = V(:); % 2I*1 matrix
    c_stacked = c(:); % 2I*1 matrix

    % A = SD
    SD_u = spdiags(If(:,1).*sf(:,1), 0, p.I, p.I)*Df + spdiags(Ib(:,1).*sb(:,1), 0, p.I, p.I)*Db; % I*I matrix
    SD_e = spdiags(If(:,2).*sf(:,2), 0, p.I, p.I)*Df + spdiags(Ib(:,2).*sb(:,2), 0, p.I, p.I)*Db; % I*I matrix
    SD = [SD_u, sparse(p.I, p.I);
         sparse(p.I, p.I), SD_e]; % 2I*2I matrix
   
    % P = A + A_switch
    P = SD + A_switch;

    % B = [(rho + 1/Delta)*I - P]
    B = (p.rho + 1/p.Delta)*speye(2*p.I) - P; 

    % b = u(c) + 1/Delta*V
    b = p.u(c_stacked) + (1/p.Delta)*V_stacked;

    % V = B\b;
    V_update = B\b; % 2I*1 matrix
    V_change = V_update - V_stacked;
    V = reshape(V_update, p.I, 2); % I*2 matrix

    % 3-6. Convergence criterion
    dist(n) = max(abs(V_change));
    if dist(n)<p.tol
       disp('Value function converged. Iteration = ')
       disp(n)
       break
    end
end

%% 5. KF EQUATION

% 5-1. Solve for 0=gdot=P'*g

PT = P';
gdot_stacked = zeros(2*p.I,1);

% need to fix one value, otherwise matrix is singular
i_fix = 1;
gdot_stacked(i_fix)=.1;

row_fix = [zeros(1,i_fix-1),1,zeros(1,2*p.I-i_fix)];
AT(i_fix,:) = row_fix;

g_stacked = PT\gdot_stacked; 

% 5-2. Normalization

g_sum = g_stacked'*ones(2*p.I,1)*da;
g_stacked = g_stacked./g_sum;

% 5-3. Reshape

gg = reshape(g_stacked, p.I, 2);

        %% 5-4. COMPUTE VARIABLES FOR A GIVEN r_r(nr)
        % Notes: Each matrix has dimensions p.I*2(u,e)*nr
        
            g_r(:,:,nr) = gg;
            adot(:,:,nr) = zz + r.*aa - c;
            V_r(:,:,nr) = V;
            dV_r(:,:,nr) = dV_upwind;
            c_r(:,:,nr) = c;
            
            S(nr) = gg(:,1)'*a*da + gg(:,2)'*a*da;
            Sp = excess_supply(nr, r, zz, aa, Df, Db, V, A_switch, da, a);
            S_prime = (Sp - S(nr))/(0.001);
            r = r - (S(nr)/S_prime)
            

        if abs(S(nr))<p.tol_S
            disp('Equilibrium Found, Interest rate =')
            disp(r)
            break
        end

end

%% 6. GRAPHS 

% 6-1. Optimal consumption 

set(gca, 'FontSize', 18)
plot(a, c_r(:,1,nr), 'LineWidth', 2, 'LineStyle', '-', 'Color', 'r')
hold on
plot(a, c_r(:,2,nr), 'LineWidth', 2, 'LineStyle', '-', 'Color', 'b')
hold off
grid
xlabel('Wealth, a','FontSize', 14)
ylabel('Consumption, c_j(a)','FontSize', 14)
xlim([p.amin p.amax])
legend(sprintf('Unemployed, r=%.4f', r), ...
       sprintf('Employed, r=%.4f', r), 'Location', 'best', 'FontSize', 14)
saveas(gcf, 'fig1.png')
% 6-2. Optimal savings 

set(gca, 'FontSize', 18)
plot(a, adot(:,1,nr), 'LineWidth', 2, 'LineStyle', '-', 'Color', 'r')
hold on
plot(a, adot(:,2,nr), 'LineWidth', 2, 'LineStyle', '-', 'Color', 'b')
hold off
grid
xlabel('Wealth, a', 'FontSize', 14)
ylabel('Saving, s_j(a)', 'FontSize', 14)
xlim([p.amin p.amax])
legend(sprintf('Unemployed, r=%.4f', r), ...
       sprintf('Employed, r=%.4f', r), 'Location', 'best', 'FontSize', 14)
saveas(gcf, 'fig2.png')
% 6-4. Wealth distribution

set(gca, 'FontSize', 14)
plot(a, g_r(:,1,nr), 'LineWidth', 2, 'LineStyle', '-', 'Color', 'r')
hold on
plot(a, g_r(:,2,nr), 'LineWidth', 2, 'LineStyle', '-', 'Color', 'b')
hold off
grid
xlabel('Wealth, a', 'FontSize', 14)
ylabel('Densities, g_j(a)', 'FontSize', 14)
yy = get(gca, 'yLim');
hold on
plot([p.amin, p.amin], yy, '--k', 'LineWidth', 2)
hold off
text(-0.15, yy(1)-0.02*(yy(2) - yy(1)), '$\underline{a}$', 'HorizontalAlignment', 'center', 'FontSize', 15, 'Interpreter', 'latex')
xlim([-0.2 1])
legend(sprintf('Unemployed, r=%.4f', r), ...
       sprintf('Employed, r=%.4f', r), 'Location', 'best', 'FontSize', 14)
saveas(gcf, 'fig3.png')


\end{lstlisting}


