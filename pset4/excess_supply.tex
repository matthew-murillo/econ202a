


% Define MATLAB code style for listings package
\definecolor{mygreen}{rgb}{0,0.6,0}
\definecolor{mygray}{rgb}{0.5,0.5,0.5}
\definecolor{mymauve}{rgb}{0.58,0,0.82}

\lstset{
    language=Matlab,
    keywordstyle=\color{blue},
    commentstyle=\color{mygreen},
    stringstyle=\color{mymauve},
    basicstyle=\ttfamily\small,
    breaklines=true,
    numbers=left,
    numberstyle=\tiny\color{mygray},
    stepnumber=1,
    numbersep=10pt,
    tabsize=4,
    showspaces=false,
    showstringspaces=false
}



\title{MATLAB Script}
\author{Matt Murillo}
\date{}

\maketitle

\section*{MATLAB Code}
\begin{lstlisting}
function S = excess_supply(nr, r, zz, aa, Df, Db, V, A_switch, da, a)

p = define_parameters_GE();

r = r + 0.001;

for n=1:p.maxit

    % 4-1. Compute the derivative of the value function 
    dVf = Df*V;
    dVb = Db*V;

    % 4-2. Boundary conditions
    dVb(1,:) = p.mu(zz(1,:) + r.*aa(1,:)); % a>=a_min is enforced (borrowing constraint)
    dVf(end,:) = p.mu(zz(end,:) + r.*aa(end,:)); % a<=a_max is enforced which helps stability of the algorithm

    I_concave = dVb > dVf; % indicator whether value function is concave (problems arise if this is not the case)

    % 4-3. Compute the optimal consumption
    cf = p.inv_mu(dVf);
    cb = p.inv_mu(dVb);
    
    % 4-4. Compute the optimal savings
    sf = zz + r.*aa - cf;
    sb = zz + r.*aa - cb;

    % 4-5. Upwind scheme
    If = sf>0;
    Ib = sb<0;
    I0 = 1-If-Ib;
    dV0 = p.mu(zz + r.*aa); % If sf<=0<=sb, set s=0

    dV_upwind = If.*dVf + Ib.*dVb + I0.*dV0;

    c = p.inv_mu(dV_upwind);

    % 4-6. Update value function: 
    % Vj^(n+1) = [(rho + 1/Delta)*I - (Sj^n*Dj^n+A_switch)]^(-1)*[u(cj^n) + 1/Delta*Vj^n]
    
    V_stacked = V(:); % 2I*1 matrix
    c_stacked = c(:); % 2I*1 matrix

    % A = SD
    SD_u = spdiags(If(:,1).*sf(:,1), 0, p.I, p.I)*Df + spdiags(Ib(:,1).*sb(:,1), 0, p.I, p.I)*Db; % I*I matrix
    SD_e = spdiags(If(:,2).*sf(:,2), 0, p.I, p.I)*Df + spdiags(Ib(:,2).*sb(:,2), 0, p.I, p.I)*Db; % I*I matrix
    SD = [SD_u, sparse(p.I, p.I);
         sparse(p.I, p.I), SD_e]; % 2I*2I matrix
   
    % P = A + A_switch
    P = SD + A_switch;

    % B = [(rho + 1/Delta)*I - P]
    B = (p.rho + 1/p.Delta)*speye(2*p.I) - P; 

    % b = u(c) + 1/Delta*V
    b = p.u(c_stacked) + (1/p.Delta)*V_stacked;

    % V = B\b;
    V_update = B\b; % 2I*1 matrix
    V_change = V_update - V_stacked;
    V = reshape(V_update, p.I, 2); % I*2 matrix

    % 3-6. Convergence criterion
    dist(n) = max(abs(V_change));
    if dist(n)<p.tol
       disp('Value function converged. Iteration = ')
       disp(n)
       break
    end
end

%% 5. KF EQUATION

% 5-1. Solve for 0=gdot=P'*g

PT = P';
gdot_stacked = zeros(2*p.I,1);

% need to fix one value, otherwise matrix is singular
i_fix = 1;
gdot_stacked(i_fix)=.1;

row_fix = [zeros(1,i_fix-1),1,zeros(1,2*p.I-i_fix)];
AT(i_fix,:) = row_fix;

g_stacked = PT\gdot_stacked; 

% 5-2. Normalization

g_sum = g_stacked'*ones(2*p.I,1)*da;
g_stacked = g_stacked./g_sum;

% 5-3. Reshape

gg = reshape(g_stacked, p.I, 2);

%% 5-4. COMPUTE VARIABLES FOR A GIVEN r_r(nr)
% Notes: Each matrix has dimensions p.I*2(u,e)*nr

    g_r(:,:,nr) = gg;
    adot(:,:,nr) = zz + r.*aa - c;
    V_r(:,:,nr) = V;
    dV_r(:,:,nr) = dV_upwind;
    c_r(:,:,nr) = c;
    
    S = gg(:,1)'*a*da + gg(:,2)'*a*da;





\end{lstlisting}


